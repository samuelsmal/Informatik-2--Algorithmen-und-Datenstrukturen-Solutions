\documentclass[10pt,a4paper]{article}
\usepackage[utf8]{inputenc}
\usepackage[german]{babel}
\usepackage{amsmath}
\usepackage{amsfonts}
\usepackage{amssymb}
\usepackage{fullpage}
\newcommand{\BigO}[1]{\ensuremath{\operatorname{O}\bigl(#1\bigr)}}
\author{Samuel von Baussnern}
\title{Solutions to exercise No. 6}
\begin{document}

\maketitle
\section{Task B: Minimum Spanning Tree}
\subsection{Jarnik, Prim and Dijkstra algorithm}

The algorithm starts with a single vertex of the graph. Then loops through all the edges of already marked vertexes and adds the edge, which satisfies two rules: 
\begin{enumerate}
\item It has the minimum weight, and 
\item wouldn't connect to an already added vertex. 
\end{enumerate}
This step is repeated until every vertex is added or no more edges satisfy the second rule.

\subsection{Practical example}

I number the vertexes of the graph clockwise (12 o'clock, 2 o'clock, 4 o'clock, etc.), for saving space I'll omit the 'o'clock'-term in the future.

The list of the added edges in chronological order starting the vertex 12:
\begin{enumerate}
\item (12, 2)
\item (2, 8)
\item (2, 6)
\item (12, 10)
\item (10, 4)
\end{enumerate}

Results in this adjacent matrix:
$
\left(
\begin{array}{cccccc}
0 & 1 & 0 & 0 & 0 & 1 \\
1 & 0 & 0 & 1 & 1 & 0 \\
0 & 0 & 0 & 0 & 0 & 1 \\
0 & 1 & 0 & 0 & 0 & 0 \\
0 & 1 & 0 & 0 & 0 & 0 \\
1 & 0 & 1 & 0 & 0 & 0
\end{array}
\right)
$

The weight of the MST: 19. The weight of the graph: 51.

\subsection{Time complexity}
The time complexitiy depends heavily on the loop through the edges of the vertexes. Therefore it's crucial to use an appropriate data structure:

\begin{table}
\begin{tabular}{l|c|c}
Data structre used & Time complexity & Reason \\
\hline \hline
Adjacency matrix & $\BigO{|V|^{2}$ & Lookup is of constant time \\
\hline
Binary heap and adjacency list & \parbox[t]{5cm}{$\BigO{(|V| + |E|) \log{|V|})}$ \\ $= \BigO{|E| \log{|V|})}$} & Finding and deleting costs $\BigO{\log{n}}$ \\
\hline
Fibonacci heap and adjacency list & $\BigO{|E| + |V| \log{|V|})}$ & \parbox[t]{5cm}{Finding is of constant cost \\ and deleting the min has a amortized cost of $\BigO{\log{n}}$}
\end{tabular}
\end{table}

The algorithm is used in computer network design, in implementing efficient circuit design, clustering gen expressions, handwriting recognition of mathematical equations and of course as a pedagogical instrument to graphs and greedy algorithms, just to name a few.

\section{Task C: Kruskal}
\subsection{Use of Kruskal's algorithm}
Since the weights are pair-wise disjoint there exists only one MST.

Listing the chosen edges:
\begin{enumerate}
\item (a,f)
\item (c,d)
\item (c,e)
\item (f,d)
\item (b,c)
\end{enumerate}

\subsection{Another example}

Six different MSTs: \\ \hfill
Listing the chosen edges:
\begin{enumerate}
\item (a,b)
\item (a,c)
\item Choice between: \begin{enumerate}
\item (a,d)
\item (c,d)
\end{enumerate}
\item Choice between: \begin{enumerate}
\item (e,d)
\item (e,b)
\end{enumerate}
\end{enumerate}

\end{document}}